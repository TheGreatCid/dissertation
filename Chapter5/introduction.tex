\section{Introduction}
\label{section: ductile/intro}

% extension to plasticity

The thermodynamic framework employed by the phase-field approach provides a convenient extension to incorporate plasticity by incorporating a plastic energy. However, how to best design the form of that plastic energy and its coupling with a phase field that regularizes a fracture surface remains an open question.
There exist many approaches to modeling plasticity in the context of phase-field fracture \cite{alessi_gradient_2014, alessi_gradient_2015, alessi_coupling_2018, ambati_phase-field_2015, ambati2016phase, miehe_phase_2016, borden2016phase, borden_phase-field_2017}.  For a recent review of the various approaches that have been proposed, see  \citet{alessi_comparison_2017}. In this chapter, we explore possible extensions to ductile fracture within the proposed variational framework of \Cref{section: Chapter2}. The proposed framework is flexible in the sense that a wide range of plastic flow rules and hardening behavior can be cast in a variational setting, as illustrated in \cite{ortiz_variational_1999}.  The consistent variational approach does impart certain restrictions, such as the degradation of the yield surface with the phase field implying a corresponding plastic contribution to crack growth.  While other models that break the variational structure may be of utility to effectively capture certain responses, they are beyond the scope of the current work.

% modeling choices

By adopting such a variational framework to model ductile fracture, the remaining major modeling choices are the form of the local term in the fracture energy (the crack geometric function)\footnote{Also sometimes referred to as the \textit{local fracture dissipation function} in the phase-field for fracture related literature, despite its energetic nature in the thermodynamic framework.}, the plastic degradation function (the functional dependence of the plastic energy on the phase-field variable), and the power expenditure. In this chapter, we extend the results to the regime of ductile fracture and demonstrate that an unperturbed elastic-plastic response can be obtained in certain cases, i.e.\ the yield stress and the plastic hardening law remain unmodified prior to crack initiation even when the plastic energy is degraded as a function of the phase-field variable.

% properties of Lorentz

In the case of brittle and quasi-brittle fracture, several studies have demonstrated that a regularization-length independent critical fracture strength and softening response can be obtained by a family of rational degradation functions in combination with the appropriate choice of the crack geometric function \cite{lorentz2011convergence, wu2017unified, geelen2019phase}. These are referred to as Lorentz-type degradation functions hereinafter.  In another work \cite{brandon2020cohesive}, we showed that this decoupling of critical fracture strength and regularization length is crucial in elastic-plastic materials to ensure the crack growth resistance is convergent with respect to the regularization. This independence or lack of sensitivity is important both in the study of the $\Gamma$-convergence of the model behavior and in engineering applications where the specimen used for calibration and the actual structure of interest span vastly different spatial scales. In this chapter, we extend the use of these degradation functions to the coupling of the plastic energy so that the model response converges as the phase-field regularization length diminishes.

Although the phase-field regularization length can be divorced from the set of material properties by choosing the Lorentz-type degradation functions, an upper bound still exists, stemming from either polyconvexity requirements or physical constraints (see e.g. \cite{wu2017unified, geelen2019phase} for detailed analysis). With regard to phase-field models for ductile fracture, various choices of the coupling between damage and plasticity can result in the need for surprisingly small regularization lengths.  In particular, some models require the need for smaller regularization lengths as the transition from plastic localization to the onset of damage is delayed.
There have been several attempts to address this issue. \citet{ambati_phase-field_2015,ambati2016phase} scaled the plastic degradation function by a critical plastic strain to be calibrated. \citet{miehe2015phase} introduced barrier functions related to critical values of the state variables. Most recently, \citet{borden2016phase} devised a threshold for the plastic work contributions to crack initiation. Although these are all valid, thermodynamically consistent approaches to introduce an arbitrary delay to crack initiation, the hybrid nature of these formulations makes it difficult to fit them into a variational framework.

Coupling the phase field to the plasticity tends to promote cracks in regions of extensive plasticity, which is often the case observed experimentally, but the observed damage process resulting from the model can be far more gradual than expected. On the other hand, when the phase field is coupled only to the elastic strain energy, the damage process is more abrupt, but the predicted crack growth directions tend not to agree with observations in ductile alloys.  In particular, they tend to propagate orthogonally to the maximum tensile stress as in brittle materials, rather than in response to intense plastic straining. Here, we have identified a plausible micro-mechanical argument to reconcile these problems, and used this to formulate a model that fractures in regions with large plastic deformation even without the strong coupling between plasticity and fracture. Motivated by findings from several recent attempts to homogenize the fracture toughness, in this work we decouple the plastic disspation from fracture degradation and propose a novel coalescence dissipation to model the coupling between plasticity and fracture. \citet{rodriguez2016silica}, \citet{chowdhury2019effects}, and \citet{vo2020molecular} found that ligament length, shape, and orientation of defects at the micro-scale influence the Mode-I fracture toughness. We postulate that the presence of plastic flow (or dislocations at the micro-scale) alters those properties related to defects, and at the continuum level, some configurational energy has been dissipated prior to crack initiation, effectively reducing the fracture toughness. We note that the idea of degrading the fracture toughness is not necessarily new, as a similar approach was recently proposed by \citet{yin2020ductile}. Importantly, by virtue of considering the coalescence dissipation in a variational framework, we show that fracture occurs in regions with large plastic deformation even without the strong coupling between plasticity and fracture.  An upshot is a significant reduction in the fracture driving energy and a corresponding relaxation of the upper bound for the regularization length.

% organization

This chapter is organized as follows. \Cref{section: Chapter5/theory/constitutive} presents the constitutive choices, including the large deformation strain energy density, plastic energy density, fracture energy density, the Fourier potential, and their corresponding dual kinetic potentials.
\Cref{section: Chapter5/verification/homogenized} motivates the choices of the crack geometric function and the degradation function by examining a homogeneous problem in detail. \Cref{section: Chapter5/verification/nonhomogeneous} demonstrates the regularization-length-independent softening behavior as well as the effect of the coalescence dissipation term.
\Cref{section: Chapter5/examples/3pb} applies the proposed model to simulate a three-point bending problem, \Cref{section: Chapter5/examples/SFC} presents simulation results for a recent Sandia Fracture Challenge problem, and \Cref{section: Chapter5/examples/spallation} shows how a creep fracture model instantiated from the variational framework can be used to model spallation of oxide scales.
