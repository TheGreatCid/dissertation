\section{Kinematics and Constraints}
\label{section: Chapter2/kinematics}

Let us start by defining degrees of freedom in the system. Let $\body$ be a body consisting of a continuous collection of material points. Let $\body_0 \subset \mathbb{R}^d$ be the reference configuration at some time $t = t_0$, whose particles are identified by their position $\btX$, and $\body_t \subset \mathbb{R}^d$ be the current configuration at a time $t > t_0$, with particles at position $\btx$. Let $\bs{\upphi}: \body_0 \times [t_0, t] \mapsto \mathbb{R}^d$ be the deformation map.
The deformation gradient is denoted as $\defgrad = \grad \bs{\upphi}$, where the operator $\grad$ denotes differentiaton with respect to $\btX$ (in the reference configuration $\body_0$). The Jacobian determinant of the deformation gradient is written as $J = \det\defgrad$.

Throughout this contribution, the \emph{local thermodynamic state} of an infinitesimal material neighborhood is defined by:
\begin{itemize}
  \item the deformation gradient $\defgrad$ from the Lie group of invertible and orientation-preserving linear transformations in $\mathbb{R}^d$;
  \item the collection of internal variables $\bfZ \in \mathbb{M}$, where the set $\mathbb{M}$ depend on the material;
  \item the entropy density per unit volume $s \in \mathbb{R}$ in the reference configuration;
\end{itemize}

Plastic (or creep) deformations are modeled through the framework of multiplicative decomposition
\begin{align}
  \defgrad = \defgrad^e \defgrad^p \defgrad^g,
\end{align}
where $\defgrad^e$, $\defgrad^p$ are referred to as the elastic and plastic deformation gradients, respectively. $\defgrad^g$ is the product of eigen deformation gradients. For example, isotropic thermal expansion can be accounted for by $\defgrad^g = (1+\alpha \Delta T)\bfI$, where $\alpha$ is the thermal expansion coefficient, and $\Delta T = T - T_0$ is the temperature change from the reference temperature $T_0$.

Crack surfaces are regularized and modeled using a phase-field $d$, $0 \leqslant d \leqslant 1$, where $d = 0$ represents the intact state of a material point, and $d = 1$ indicates the material point has lost all of its load-carrying capacity.

For the systems of interest in this contribution, the internal variables are considered to be the plastic deformation $\bfF^p$ (from the same Lie group of $\bfF$), the (scalar) effective plastic strain $\ep \in \mathbb{R}_+$, and the phase-field $d$ regularizing the crack surfaces, i.e.
\begin{align}
  \bfZ = \left\{ \bfF^p, \ep, d \right\}.
\end{align}
In general, internal variables evolve subject to constraints of the following form
\begin{align}
  \bfL(\bfZ)\dot{\bfZ} = \bs{0}. \label{eq: general internal variable constraint}
\end{align}
The specific form of the constraint depends on physical requirements and the material class. For isotropic $J_2$-plasticity, the Prandtl-Reuss flow rule requires
\begin{subequations}
  \begin{align}
    \tr\left( \dot{\bfF}^p {\bfF^p}^{-1} \right)                     & = 0, \label{eq: prandtl-reuss isochoric} \\
    \norm{\dot{\bfF}^p {\bfF^p}^{-1}}^2 - \dfrac{3}{2}\abs{\epdot}^2 & = 0, \label{eq: prandtl-reuss uniaxial}  
  \end{align}
\end{subequations}
where \eqref{eq: prandtl-reuss isochoric} requires the plastic flow to be isochoric, and \eqref{eq: prandtl-reuss uniaxial} normalizes the effective plastic strain $\ep$ to be uniaxial. Furthermore, the plastic flow is assumed to be irreversible in the sense
\begin{align}
  \epdot \geqslant 0. \label{eq: plastic irreversibility}
\end{align}

Since the phase-field variable $d$ is an approximation to the irreversible crack set, and as will be shown later, the crack geometric function is a monotonically increasing function of $d$, it is intuitive to impose the irreversibility constraint on the phase-field variable, i.e.
\begin{align}
  \dot{d} \geqslant 0. \label{eq: fracture irreversibility}
\end{align}
Note that \eqref{eq: prandtl-reuss isochoric}, \eqref{eq: prandtl-reuss uniaxial}, \eqref{eq: plastic irreversibility}, and \eqref{eq: fracture irreversibility} are all of the general form \eqref{eq: general internal variable constraint}.
