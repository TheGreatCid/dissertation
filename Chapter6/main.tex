\chapter{Conclusions}
\label{section: Chapter6}

In this dissertation, the variational approach to fracture is extended to account for general dissipative solids and different types of fracture. Several models are constructed within the proposed variational framework to study practical engineering problems.

In \Cref{section: Chapter3}, a model for brittle/quasi-brittle fracture is developed to investigate fracture evolution at the microscale. 3D numerical studies are performed to investigate the effect of bubble geometry, porosity and loading conditions on the critical fracture strength. The bubble geometry affects the fracture stress by changing the stress concentration near the bubble and grain boundary interface. Loading conditions also make an important impact on the critical fracture strength -- with multiple loading directions, the critical fracture strength appears to be lower. Intergranular crack propagation simulations have been performed considering different porosity levels. The variation of the critical fracture strength with porosity has then been compared with experimental measurements. 3D simulation results show a better agreement with the experiment data than the previous 2D results. The model is then extended to account for pressurized cracks. The model is proven to be capable of reflecting fine fragmentation phenomena and providing insights into the mechanisms of HBS fine fragmentation. Simulations show that fracture occurs along the grain boundaries surrounding the large fission gas bubbles, which agrees with the hypothesized mechanism of fine fragmentation during LOCA transients. Bubble size and the amount of external pressure will change the critical pressure for crack initiation. Crack propagation at different recrystallization stages on a partial HBS is also investigated using this model. The recrystallization changes the grain structure, thus altering crack propagation paths and directions.

In \Cref{section: Chapter4}, a model for cohesive fracture is used to study soil dessication. A novel split of the elastic potential is proposed to account for the external work done by contact forces on regularized crack surfaces. Such a split results in a constitutive relation that naturally enforces traction-free boundary conditions and contact conditions in the vicinity of the cracks. The advantage of the contact split in systems subject to shear loading or even purely tensile loading is demonstrated in \Cref{section: Chapter4/verification}. We then develop a stochastic model for the random field of fracture properties in \Cref{section: Chapter4/theory/stochastic}. The representation enables the integration of pointwise correlation between the fracture toughness and the critical fracture energy, and two prototypical covariance models are specifically considered to offer modeling flexibility.
The variational phase-field fracture model with the proposed contact split is then applied to study thin-film fracture. A qualitative agreement with an analytical solution is established using a quasi one-dimensional simplification. A two-dimensional plane-stress model is subsequently used to explore the effect of random fracture properties on crack morphology. Forward propagation problems are first considered to study the effect of kernel smoothness, correlation length, and pointwise correlation between the fracture properties. In particular, it is observed that (1) smooth fracture properties result in larger fragment sizes (in mean); (2) the crack path is mainly influenced by the spatial variations of the fracture toughness, rather than fluctuations in the critical fracture energy; and (3) different compositions of covariance functions can result in dramatically different crack patterns. Finally, the versatility offered by the probabilistic framework is highlighted by solving a statistical inverse problem based on physical experiments. Despite the apparent simplicity of the identification methodology, it is seen that the stochastic phase-field formulation allows key characteristics of observed crack patterns to be reproduced with reasonable accuracy, hence paving the way for future developments coupling fracture simulations with experimental results exhibiting substantial variability.

In \Cref{section: Chapter5}, several models for ductile fracture are constructed within the variational framework. The framework provides two mechanisms for coupling  plasticity and fracture, either using the plastic degradation function (E-P-PD) or through the coalescence dissipation (E-P-D). The proposed E-P-D model using this novel coalescence dissipation was compared with the E-P-PD model from \cite{brandon2020cohesive}.
Several important modeling choices are discussed in \Cref{section: Chapter5/verification/homogenized} in a ``homogenized'' setting:
A crack geometric function $\alpha(d; \xi > 0)$ allows for an unperturbed elastic-plastic response prior to damage initiation (\Cref{section: Chapter5/verification/homogenized/alpha});
a Lorentz-type degradation function separates out the critical fracture energy as an independent material property such that a regularization-length independent finite critical fracture strength can be obtained (\Cref{section: Chapter5/verification/homogenized/degradation});
the decrease in the fracture toughness as a function of the increasing plastic strain can be modeled using the proposed coalescence dissipation (\Cref{section: Chapter5/verification/homogenized/coalescence}).
Convergence of the softening response is numerically demonstrated in \Cref{section: Chapter5/verification/nonhomogeneous}.
The force-displacement responses obtained using both the proposed E-P-D model and the E-P-PD model presented in \cite{brandon2020cohesive} become insensitive to the regularization length as the regularization length is sufficiently small, regardless of the amount of coalescence dissipation.
The performance of the proposed E-P-D model is evaluated by simulating a three-point bending experiment in \Cref{section: Chapter5/examples/3pb}. The model is first calibrated against a standard tension test of a notched specimen, and then applied to predict the crack path and load-deflection response of the aluminum alloy specimen in the three-point bending experiment.
The predicted load-deflection curves and crack trajectories are found to be in good agreement with experimental measurements. Finally, the capabilities of the proposed E-P-PD model are examined by simulating an experiment from a recent Sandia Fracture Challenge. Both the predicted force-displacement response and the crack path indicate an excellent agreement with the experimental observations.

% Advantages of the variational framework

Besides the advantages from theoretical aspects mentioned in \Cref{section: Chapter1}, several other advantages of the variational framework from a modeling perspective can be summarized from the research presented in \Cref{section: Chapter3,section: Chapter4,section: Chapter5}:
the framework is flexible enough to describe a wide range of coupled field phenomena and many common fracture mechanisms; the definitions of specific stress-strain relations, micro- and macro-force balances, plastic flow, yield surface, heat generation and boundary conditions all follow directly from physically motivated constitutive energies and potentials; and the restrictions imposed on the constitutive choices to ensure existence and uniqueness of the solution also lead to numerically well-behaved models.

% Future work

There are other coupled field phenomena that have not been incorporated into the proposed variational framework in its current form. What follows are some future research directions.
\begin{itemize}
  \item Fatigue effects are important in structures subject to cyclic loading. There has been a number of attempts to incorporate fatigue effects in the phase-field model for fracture \cite{seiler2020efficient,mesgarnejad2019phase,alessi2018phenomenological,lo2019phase}, all of which introduce fatigue effects by ``degrading'' the fracture toughness as a function of certain fatigue measure. Casting these models into the proposed variational framework requires the construction of a dual kinetic potential representing configurational changes of the material (similar to the coalescence dissipation introduced in \Cref{section: Chapter5}).
  \item The proposed variational framework is suitable for modeling fracture propagation which is mainly driven by energetics. The fracture driving energies are often decomposed to provide some ad-hoc strength envelope for fracture nucleation. However, several authors have found that the existing decompositions of the strain energy density are not sufficiently flexible to model general strength envelopes, e.g. the Drucker-Prager criterion \cite{tanne2018crack,kumar2020revisiting}. \citet{kumar2020revisiting} introduced a micro body force blended with the crack geometric function to model general strength envelopes for nucleation. Recently, \citet{de2021nucleation} proposed a variational formulation for nucleation which can be fit within our proposed variational framework. How best to extend such formulation to model the transition from nucleation to propagation remains to be an open question.
  \item As a direct extension of the work presented in \Cref{section: Chapter5}, the ductile fracture model can be applied to simulate ductile failure with dynamic effects, e.g. subject to impact loading. For example, the Kalthoff-Winkler experiment \cite{kalthoff1988failure} shows two qualitatively different failure mode at low and high loading rates. At low impact loading rates, brittle fracture is observed, which has become a standard benchmark problem for phase-field models of brittle fracture. At high impact loading rates, a shear band is formed due to large plastic deformation. There has been few attempts to simulate the ductile failure at high loading rates. \cite{ulmer2013phase} is a notable exception, where two different sets of parameters are calibrated for the two failure modes. A model which can simulate the two failure modes with the same set of parameters is yet to be discovered.
  \item Stress triaxiality effects are important in ductile failure. ``Shear lips'' are often observed in regions with high stress triaxiality, e.g. \Cref{fig: Chapter5/3pb/split_comparison}. \citet{borden2016phase} proposed a means of incorporating stress triaxiality effects by scaling the plastic work. \citet{ambati_phase-field_2015,ambati2016phase} scaled the plastic degradation function by a critical plastic strain to be calibrated. Although not variational, these approaches are shown to be capable of capturing shear lips. It would be of interest to incorporate stress triaxiality effects within the variational framework.
\end{itemize}
