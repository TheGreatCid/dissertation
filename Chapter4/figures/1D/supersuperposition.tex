\begin{figure}[!htb]
  \centering
  \begin{overpic}[scale=0.55]{Chapter4/figures/1D/supersuperposition.png}
    \put(12,-4){(a)}
    \put(46,-4){(b)}
    \put(80,-4){(c)}
  \end{overpic}
  \vspace{0.35in}
  \caption{ The principal of ``super-superposition'' applied to the region of interest marked in \Cref{fig: Chapter4/1D/schematic}. The analytical solution is derived based on the boundary conditions shown in (c). }
  \label{fig: Chapter4/1D/supersuperposition}
\end{figure}
