\section{Research contributions}

%% summary
This dissertation aims to extend the existing variational framework to account for thermal-mechanical-fracture coupling. The proposed framework is able to account for large deformation, inelastic deformation, thermal effects (i.e. heat conduction, heat convection, and heat generation), viscous effects, and potentially dynamic effects. From a fracture mechanics point of view, the framework is flexible enough to model brittle, quasi-brittle, cohesive, and ductile fracture. The applicability to each type of fracture is examined in the following chapters: A brittle/quasi-brittle instantiation is applied to model three-dimensional intergranular fracture and fission-gas-induced fragmentation in microstructures, a cohesive fracture model is applied to revist the soil dessication problem with a focus on random fracture properties, and a family of ductile fracture models are developed to simulate quasi-static ductile failure under isothermal conditions as well as thermal-induced oxide spallation.

\begin{itemize}
  \item %% contributions in chapter 3: 3D intergranular fracture, fission-gas-induced fragmentation
        In \Cref{section: Chapter3}, phase-field for brittle/quasi-brittle fracture models are employed to simulate intergranular cracking with multiple bubbles on the grain boundaries in 3D and over-pressurized fission-gas-induced fragmentation at the microstructural level. To account for pressurized crack surfaces, we present an extension of the quasi-brittle model by including regularized external work done by pressure into the total energy of the system. The regularized external work is derived based on a phase-field approximation of the sharp interface. We demonstrate that the prediction of the dependence of the critical fracture strength on porosity based on 3D simulations is better than the 2D prediction. The effects of multi-bubble interaction and partial recrystallization in HBS are also investigated.
  \item %% contributions in chapter 4: contact split, interfacial energy, random properties
        In \Cref{section: Chapter4}, a phase-field for cohesive fracture model is used to simulate pervasive cracking in thin films. A new strain energy split is proposed to enforce frictionless contact conditions in the vicinity of diffuse fracture surfaces. In contrast to existing splits that have been proposed, our approach completely prevents tractions from being transmitted across fully damaged surfaces that are loaded in tension. We construct a probabilistic model for the critical fracture energy and the fracture toughness, modeled as (potentially correlated) random fields, and demonstrate, through forward analysis and by solving an inverse problem, how crack network morphology can be influenced by stochastic spatially-varying material properties.
  \item %% contributions in chapter 5: variational ductile fracture, coalescence dissipation, thermal effects
        In \Cref{section: Chapter5}, we present a variational model for ductile fracture, and demonstrate, through numerical examples, that the model has the following properties: An unperturbed elastic-plastic response can be obtained and the plastic hardening law remain unmodified prior to crack initiation even when the plastic energy is degraded as a function of the phase-field variable; a regularization-length independent critical fracture strength and softening response can be obtained by a family of rational degradation functions; an novel dual kinetic potential, termed with \emph{the coalescence dissipation}, can be incorporated to introduce an alternative form of coupling between plasticity and fracture. Thermal effects are then incorporated to model spallation of the oxide scale in a high-temperature heat exchanger.
\end{itemize}
