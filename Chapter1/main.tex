\chapter{Introduction}
\label{section: Chapter1}

Fracture can be observed in virtually all engineering applications across several scales. It is important to understand the behavior of materials, components and structures with the presence of fracture. To that end, model-based simulations are widely adopted to help understand fracture-related phenomena. In the past two decades, variational approaches to fracture have become increasingly popular, among which gradient damage models and phase-field models have received the most attention. On the one hand, these models have been successfully applied to a variety of engineering problems involving brittle/quasi-brittle fracture, cohesive fracture, ductile fracture as well as their coupling with other physics. On the other hand, the development of the theoretical variational framework was mostly confined within the regime of brittle/quasi-brittle fracture. This dissertation is concerned with the gap between the fundamental variational framework and practical engineering applications.

\section{Background}

%% Fracture mechanics basics
In this dissertation, four types of fracture are of interest: brittle fracture, quasi-brittle fracture, cohesive fracture, and ductile fracture. Brittle and quasi-brittle fracture center around the Griffith-type description of fracture, in the sense that the stress field at the crack tip is singular. In particular, brittle fracture is characterized with the complete loss of tensile strength upon crack nucleation. Under displacement-controlled uni-axial tension, a brittle specimen forms a through crack immediately after crack nucleation. Quasi-brittle fracture, in contrast, shows a gradual decay of tensile stress which is often measured in the form of a traction-separation law. Although the celebrated Griffith theory of fracture captures the behavior of wide range of brittle and quasi-brittle materials, it bears emphasis that most materials are not perfectly brittle in the Griffith sense. Cohesive fracture, for example, describes materials that possess a \emph{fracture process zone} ahead of the crack tip. The fracture process zone (FPZ) is a lumped description of small-scale yielding, void coalescence, and micro-crack merging. Typically, the material inside the FPZ only partially loses its load-carrying capacity, and therefore cohesive fracture also displays a gradual decay of tensile stress after crack initiation. The boundary between the definitions of quasi-brittle fracture and cohesive fracture is inherently vague. In this dissertation, cohesive fracture is assumed to have a substantially larger fracture process zone than quasi-brittle fracture, and at the engineering scale, numerical models for these two types of fracture are often used interchangeably. Lastly, ductile fracture describes the softening due to fracture after the stage of plastic hardening. Most metals and fiber-reinforced cementitious composites are ductile and are subject to ductile failure,

%% Briefly mention other methods to model fracture
A number of models have been proposed to model fracture. In what follows is a brief, non-comprehensive survey of numerical methods to model fracture. Particle-based and mesh-free discretization techniques, such as discrete element method (DEM), lattice spring methods, peridynamics theory, have been developed for fracture modeling. The key advantage of these methods lies in the fact that cracks can nucleate and grow in an unguided manner. However, the computational cost of these methods can be very high and the interface conditions are not straightforward to apply. Within the context of the finite element method (FEM), linear elastic fracture mechanics models are extensively used, among which the most commonly used methods are the extended finite element method (XFEM) and the cohesive zone method (CZM). In XFEM, cracks are represented as discrete discontinuities and the displacements with discontinuities are enriched using the partition of unity method \cite{babuaka1997, Dolbow99}. Singularities associated with fracture can be represented using XFEM enrichment at relatively low costs. However, tracking the evolution of complex fracture surfaces, particularly in 3D, is quite tedious. In CZM, damage evolution is governed by a traction-separation law at crack surfaces \cite{needleman_1992, ortiz_1999}. For polycrystalline material, CZM can explicitly account for grain boundaries and model intergranular crack initiation and evolution \cite{KAMAYA2007, KAMAYA2009}. Nonetheless, CZM suffers from mesh dependency because the representation of crack propagation is restricted to element boundaries.

%% Origin of phase-field fracture models and how it works
The variational approach to fracture was proposed by \citet{Francfort98}, and the phase-field implementation is attributed to \citet{Bourdin2000}.
\citet{bourdin2008variational} provides an overview.  In the variational approach, crack surfaces are represented by a surface density function in terms of an auxiliary phase-field. This naturally gives rise to a regularization involving a length scale parameter. Recent works have illustrated potential of phase-field models to be predictive for a wide range of fracture problems. Phase-field for fracture models have succeeded in capturing complex crack patterns, including branching and merging in both two and three dimensions \cite{karma_2001, karma_2004, henry_2004, spatschek_2007, amor_2009}.

%% Seminal works on the variational framework
The general variational framework was first proposed by \citet{biot1956thermoelasticity} in the context of thermoelasticity, followed by extensive investigations in coupled thermoelastic and thermoviscoelastic systems \cite{herrmann1963variational,ben1965variational,oden2012variational,molinari1987global,batra1989principle,matsubara2021variationally}. Well-defined variational principles for dissipative solids with and without heat conduction also exist \cite{ortiz_1999,yang2006variational}. However, the case of thermo-mechanical-fracture coupling in dissipative materials has received far less attention.

%% What variational really means
It is worth noting that there are fundamental differences between variational consistency and thermodynamic consistency. In fact, thermodynamically consistent models are not necessarily variational. To satisfy thermodynamics, the model \emph{must} satisfy the mass conservation, linear momentum conservation, angular momentum conservation, as well as the first and the second laws of thermodynamics. Violating any of the aforementioned conservations or thermodynamic laws results in a physically inadmissible model. However, since the second law of thermodynamics is an inequality statement, there are infinitely many ways to satisfy it when modeling a dissipative solid. In contrast, the variational approach (including, but not limited to, variational approaches to fracture) is based on the idea that, upon careful construction of a potential, its variational statements are thermodynamically consistent, implying that the dissipation attains its maximum at the solution corresponding the variational statements. In other words, the variational approach sets a more restrictive stage for model construction because it places more constraints on the constitutive behavior of the material. For example, if a potential is constructed within the variational framework to account for thermomechanical coupling, the heat generation due to its dissipation is directly predicted by the variational statement and cannot be specified arbitrarily.

%% Benefits of a variational model
The restrictions imposed by the variational approach lead to several beneficial aspects. First, the variational framework enables application of the tools of calculus of variations to the analysis of the solutions of the problem. In particular, direct method of calculus of variations informs conditions for the existence and uniqueness of solutions (e.g. \cite{dal2012introduction}). Second, localization of fracture and plastic zone can be studied within the framework of free-discontinuity problems \cite{braides1998approximation,gariepy2001functions}. Third, in numerical methods, discretization of the variational statement of the problem leads to a symmetric operator, which can save storage and potentially accelerates the assembly process. Furthermore, energy-based line search methods can be directly applied to the system of equations; many powerful numerical optimization packages (e.g. PETSc \cite{petsc-web-page}, TAO \cite{benson2003tao}, Trilinos \cite{heroux2005overview}, and Matlab \cite{higham2016matlab}) can be utilized out-of-the-box; the time-discretized variational problem leads to robust and efficient variational constitutive update algorithms \cite{ortiz_1999}.


\section{Research contributions}

%% summary
This dissertation aims to extend the existing variational framework to account for thermal-mechanical-fracture coupling. The proposed framework is able to account for large deformation, inelastic deformation, thermal effects (i.e. heat conduction, heat convection, and heat generation), viscous effects, and potentially dynamic effects. From a fracture mechanics point of view, the framework is flexible enough to model brittle, quasi-brittle, cohesive, and ductile fracture. The applicability to each type of fracture is examined in the following chapters: a brittle/quasi-brittle instantiation is applied to model three-dimensional intergranular fracture and fission-gas-induced fragmentation in microstructures; a cohesive fracture model is applied to revisit the soil desiccation problem with a focus on random fracture properties; and a family of ductile fracture models are developed to simulate quasi-static ductile failure under isothermal conditions as well as thermal-induced oxide spallation.

\begin{itemize}
  \item %% contributions in chapter 3: 3D intergranular fracture, fission-gas-induced fragmentation
        In \Cref{section: Chapter3}, phase-field for brittle/quasi-brittle fracture models are employed to simulate intergranular cracking with multiple bubbles on the grain boundaries in 3D and over-pressurized fission-gas-induced fragmentation at the microstructural level. To account for pressurized crack surfaces, we present an extension of the quasi-brittle model by including regularized external work done by pressure into the total energy of the system. The regularized external work is derived based on a phase-field approximation of the sharp interface. We demonstrate that the prediction of the dependence of the critical fracture strength on porosity based on 3D simulations is better than the 2D prediction. The effects of multi-bubble interaction and partial recrystallization in HBS are also investigated.
  \item %% contributions in chapter 4: contact split, interfacial energy, random properties
        In \Cref{section: Chapter4}, a phase-field for cohesive fracture model is used to simulate pervasive cracking in thin films. A new strain energy split is proposed to enforce frictionless contact conditions in the vicinity of diffuse fracture surfaces. In contrast to existing splits that have been proposed, our approach completely prevents tractions from being transmitted across fully damaged surfaces that are loaded in tension. We construct a probabilistic model for the critical fracture energy and the fracture toughness, modeled as (potentially correlated) random fields, and demonstrate, through forward analysis and by solving an inverse problem, how crack network morphology can be influenced by stochastic spatially-varying material properties.
  \item %% contributions in chapter 5: variational ductile fracture, coalescence dissipation, thermal effects
        In \Cref{section: Chapter5}, we present a variational model for ductile fracture, and demonstrate, through numerical examples, that the model has the following properties: a unperturbed elastic-plastic response can be obtained and the plastic hardening law remain unmodified prior to crack initiation even when the plastic energy is degraded as a function of the phase-field variable; a regularization-length independent critical fracture strength and softening response can be obtained by a family of rational degradation functions; and a novel dual kinetic potential, termed with \emph{the coalescence dissipation}, can be incorporated to introduce an alternative form of coupling between plasticity and fracture. Thermal effects are then incorporated to model spallation of the oxide scale in a high-temperature heat exchanger.
\end{itemize}


\section{Organization of the dissertation}

First, the variational framework is presented in \Cref{section: Chapter2}. General kinematics and constraints assumed throughout this dissertation are summarized in \Cref{section: Chapter2/kinematics}. \Cref{section: Chapter2/thermodynamics} provides a brief review of thermodynamic conservations and laws in their global and local forms.
The proposed variational framework is presented in \Cref{section: Chapter2/minimization}, and discretization of the governing equations is presented in \Cref{section: Chapter2/discretization}.

The variational framework is applied to solve engineering problems in \Cref{section: Chapter3,section: Chapter4,section: Chapter5}. \Cref{section: Chapter3} simulates intergranular fracture and fission-gas-induced fracture in microstructures. \Cref{section: Chapter4} revisits the soil desiccation problem and explores the effect of random fracture properties. \Cref{section: Chapter5} benchmarks the ductile fracture model with a three-point bending experiment and one of the Sandia Fracture Challenges, and models oxide spallation in high-temperature heat exchangers. Each chapter begins with a general introduction to provide context and an overview of state-of-the-art methods and models. Then, the specific constitutive choices are made, and the resulting governing equations are presented. Each chapter is concluded with verifications of the variational model and some concrete numerical examples.


\section{Notation}

In what follows, deterministic scalar, vectors, second-order tensors, and fourth-order tensors are denoted by $a$ (or $A$), $\bta$ (or $\btA$), $\bfA$, and $\mathbb{A}$, respectively.

Let $\body$ be a collection of points $\btX \in \mathbb{R}^d$, $d \in \{1, 2, 3\}$. Scalar- and vector-valued random fields defined on the probability space $(\Theta, \Sigma, P)$, indexed by $\body$, are denoted as $\{ A(\btX), \btX \in \body \}$ and $\{ \btA(\btX), \btX \in \body \}$, respectively.
At any fixed material point $\btX \in \body$, $a(\btX)$ and $\bta(\btX)$ are random variables defined on the probability space $(\Theta, \Sigma, P)$. For any fixed $\theta \in \Theta$, $a(\theta)$ and $\bta(\theta)$ are realizations of the random variables.
Similarly, $\btX \mapsto a(\btX;\theta)$ and $\btX \mapsto \bta(\btX;\theta)$ are realizations of the random fields $\{ A(\btX), \btX \in \body \}$ and $\{ \btA(\btX), \btX \in \body \}$.

Einstein summations are assumed wherever applicable unless otherwise stated. For any vectors $\bta$ and $\btb$ of the same size, the inner product is defined as $\bta \cdot \btb = a_ib_i$ where $a_i$ and $b_i$ are components of the vectors. The associated vector norm is $\norm{\bta}^2 = \bta \cdot \bta$. Similarly, for any second-order tensors $\bfA$ and $\bfB$ of the same size, the inner product is defined as $\bfA : \bfB = \tr(\bfA^T \bfB)$. The associated Frobenius norm reads $\norm{\bfA} = \sqrt{\bfA : \bfA}$. Other matrix norms will be distinguished by subscripts.
The outer (cross) product of two vectors is written as $\bta \otimes \btb = a_ib_j$. The time derivative is denoted by an over-dot, e.g. $\dot{a}$.

Macaulay brackets are denoted by triangle brackets $\macaulay{a}_\pm$ and are defined as $\macaulay{a}_\pm \equiv (a\pm\abs{a})/2$. Partial derivative is denoted by a subscript starting with a comma, i.e. $ a_{,bc} \equiv \partial^2 a / \partial b \partial c $.

