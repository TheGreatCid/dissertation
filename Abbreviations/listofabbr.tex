\abbreviations

% You can put here what you like, but here's an example
%Note the use of starred section commands here to produce proper division
%headers without bad '0.1' numbers or entries into the Table of Contents.
%Using the {\verb \begin{symbollist} } environment ensures that entries are
%properly spaced.

\section*{Symbols}

% Put general notes about symbol usage in text here.  Notice this text is
% double-spaced, as required.

\begin{symbollist}
  \item[$\btX$] A point in the reference configuration.
  \item[$\mathbb{R}$] The real line.
  \item[$d$] The spatial dimensionality.
  \item[$\Theta$] A sample space.
  \item[$\Sigma$] An event space.
  \item[$\mathbb{P}$] A probability function.
  \item[$\theta$] A realization.
  \item[$\body$] A body consisting of a continuous collection of material points.
  \item[$\body_0$] The reference configuration.
  \item[$t$] Time.
  \item[$t_0$] The initial time.
  \item[$\body_t$] The current configuration.
  \item[$\btx$] A point in the current configuration.
  \item[$\btu$] Displacement.
  \item[$\bs{\upphi}$] Deformation map.
  \item[$\defgrad$] Deformation gradient.
  \item[$J$] Determinant of the deformation gradient.
  \item[$\bfZ$] The set of internal variables.
  \item[$\mathbb{M}$] The space where the internal variables are defined.
  \item[$s$] The entropy density per unit volume.
  \item[$\defgrad^e$] Elastic deformation gradient.
  \item[$\defgrad^p$] Plastic (creep) deformation gradient.
  \item[$\defgrad^g$] Eigen deformation gradient.
  \item[$\bar{\alpha}$] The (instantaneous) coefficient of thermal expansion.
  \item[$T$] The external temperature.
  \item[$\bfI$] The second-order identity tensor.
  \item[$T_0$] The reference temperature.
  \item[$d$] Phase field.
  \item[$\ep$] Effective plastic (creep) strain.
  \item[$\rho_0$] Density in the reference configuration.
  \item[$\btb$] Body force per unit mass.
  \item[$\bfP$] The first Piola-Kirchhoff stress.
  \item[$\normal_0$] Exterior normal in the reference configuration.
  \item[$u$] The internal energy density.
  \item[$k$] The kinetic energy density.
  \item[$\mathcal{P}^\text{ext}$] The external power expenditure.
  \item[$q$] Heat source per unit mass.
  \item[$\bth$] Heat flux.
  \item[$\mathcal{S}$] The set of state variables.
  \item[$\mathcal{K}$] The set of kinematic degrees of freedom.
  \item[$\mathcal{V}$] The set of velocities.
  \item[$\mathcal{\dot{\Lambda}}$] The set of generalized velocities.
  \item[$\mathcal{P}^\text{int}$] The internal power expenditure.
  \item[$\psi$] The Helmholtz free energy.
  \item[$\delta$] The internal dissipation density.
  \item[$\bfT$] The thermodynamic conjugate of $\psi$ with respect to $\defgrad^p$.
  \item[$Y$] The thermodynamic conjugate of $\psi$ with respect to $\ep$, or the flow stress.
  \item[$f$] The thermodynamic conjugate of $\psi$ with respect to $d$.
  \item[$\bs{\xi}$] The thermodynamic conjugate of $\psi$ with respect to $\grad d$.
  \item[$\mathcal{F}^\text{eq}$] The set of equilibrium generalized forces.
  \item[$\mathcal{F}^\text{vis}$] The set of viscous generalized forces.
  \item[$c_v$] The heat capacity per unit mass.
  \item[$\delta_T$] The dissipation density accounting for the thermal effects.
  \item[$\psi^e$,$\psi^p$,$\psi^f$,$\psi^h$] The strain, plastic, fracture and thermal energy density.
  \item[${\psi^e}^*$,${\psi^p}^*$,${\psi^h}^*$] The elastic, plastic and fracture dual kinetic potentials.
  \item[$\btt$] Surface traction.
  \item[$\bar{h}_n$] The prescribed heat flux.
  \item[$h$] The heat transfer coefficient.
  \item[$T^\text{eq}$] The equilibrium temperature.
  \item[$L$] The total potential.
  \item[$\varphi$] The local potential density.
  \item[$\Delta^*$] The sum of dual kinetic potentials.
  \item[$\chi$] The Fourier potential.
  \item[$\btg$] The normalized temperature gradient.
  \item[$\bta$] Acceleration.
  \item[$\btu_g$] The prescribed displacement.
  \item[$T_g$] The prescribed temperature.
  \item[$\btu_0$] The initial displacement.
  \item[$\btv_0$] The initial velocity.
  \item[$d_0$] The initial phase field.
  \item[$\bar{T}_0$] The initial temperature.
  \item[$\mathcal{U}_t$,$\mathcal{D}_t$,$\mathcal{T}_t$] Trial spaces for the displacement, the phase field, and the temperature.
  \item[$\mathcal{V}$,$\mathcal{C}$,$\mathcal{E}$] Weighting spaces for the displacement, the phase field, and the temperature.
  \item[$\strain$] The infinitesimal strain.
  \item[$\lambda$] Lame's first parameter.
  \item[$G$] Shear modulus.
  \item[$\stress$] The Cauchy stress.
  \item[$\Gamma$] The permanent crack set.
  \item[$\Gc$] Fracture toughness.
  \item[$\gamma$] Crack surface density.
  \item[$l$] Phase-field regularization length.
  \item[$c_0$] The normalization constant for the crack surface density.
  \item[$\alpha$] The crack geometric function.
  \item[$\psi_c$] Critical fracture energy.
  \item[$\xi$] Derivative of the crack geometric function at $d = 0$.
  \item[$a_1$,$a_2$,$a_3$,$p$, $m$] Parameters in the rational degradation function.
  \item[$D_u$] The ultimate phase-field half bandwidth.
  \item[$\sigma_c$] Critical fracture strength (or ultimate strength).
  \item[$\bar{p}$] The prescribed pressure on the crack surface.
  \item[$I$] The phase-field indicator function.
  \item[$\widetilde{p}$] The effective pressure approximated by the phase field.
  \item[$E$] Young's modulus.
  \item[$\nu$] Poisson's ratio.
  \item[$w$] Separation between crack surfaces.
  \item[$h^e$] Element size.
  \item[$p_c$] The critical pressure for crack propagation.
  \item[$r$] Porosity.
\end{symbollist}

\section*{Abbreviations}

% Long lines in the \texttt{symbollist} environment are single spaced, like in
% the other front matter tables.

\begin{symbollist}
  \item[AR] Aqua Regia, also known as hydrocloric acid plus a splash of
  nitric acid.
  \item[SHORT] Notice the change in alignment caused by the label width
  between this list and the one above.  Also notice that this multiline
  description is properly spaced.
  \item[OMFGTXTMSG4ME] Abbreviations/Symbols in the item are limited to
  about a quarter of the textwidth, so don't pack too much in there.
  You'll bust the margins and it looks really bad.
\end{symbollist}
